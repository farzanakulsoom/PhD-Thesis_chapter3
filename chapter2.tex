\chapter{Overview of Wireless Communication}
\textcolor{black}{Wireless channels may be classified according to different  characteristics of propagation environments, i.e., the media in which the radio signal can propagate in the form of electromagnetic waves.} Different types of propagation channels are indoor, outdoor (urban, suburban), underwater, orbital propagation environments or any other medium in which signals propagate like, for example, the human body in intra-body communication(IBC).
%In fact   any medium where signal can propagate in a distinct fashion can be defined as channel , for example, intra body communication or snowpack has its own distinct set of challenges in terms of wireless communication.
In order to model  and define aforesaid channels , channel models are build with attributes taken from real time calculations. \\
To develop better understanding, it is essential to know that what kind of effects the channels will exert on the signal therfore, the radio wave propagation modelling is studied.
The radio waves which are generated by local disturbance in the electric and magnetic fields, spread out in all direction with distance. 
The propagation of these waves depend on medium characteristics. For example, if the propagation medium is homogeneous(i.e.  air), it will spread out uniformly in all directions and the power of the signal is attenuated with distance. However, if the medium has some sort of obstacle such as buildings, trees, mountains etc, it can undergo several attenuating effects such as reflection, refraction and scattering;  \\
-	reflection, when a wave collides any object and get reflected either completely back or at tilted angle, this changes the phase of the signal as well as it magnitude. \\
-	refraction, this effect is experienced when the wave travels from one medium to another. Refraction effects frequency and phase of the wave.\\
-	Scattering, when the wave collides at the edges of very large object the amplitude and phase of the wave is change, single wave can results into multiple waves. 

The aforesaid channel effects produce multiple paths in the channel each having its own amplitude and phases. These multiple copies of the signal interfere with each other when received on the receiver also known as multipath fading. 
%. When the distance between the transmitter and DF array is large, the contribution of these error is limited. Hence, gross errors are observed mostly due to the near-field multipath components.
There are two main categories of fading, large scale fading and small scale fading.
\begin{itemize}
    \item In large scale fading the averaged signal-power is attenuated by traveling over a long distance known as \textit{path loss} and it has larger scatters(buildings) in its paths causing a shadowing zone.
    \item Small scale fading is due to rapid fluctuations in signal amplitude or phase or because of superposition of multipath components which cause changes in amplitude and phase over a short period of time(in seconds).
    Small scale fading can be further divided into two, based on delay spread(channel response to a short pulse on multipath) and Doppler spread(frequency range over which the received doppler spectrum must be non-zero).Different types of small scale fading  is as shown in figure \ref{lb_sm_fadd}.
   
\end{itemize}
\begin{figure}
\scalebox{0.72}{
\begin{forest}
  for tree={
    font=\small\sffamily,
        draw, rounded corners,
        text centered,
        minimum height=3ex,
        text width=29ex,
        inner sep=0.5ex,
        anchor=north,
        rounded corners,
        top color=white,
        bottom color=white!20,
    l sep=5mm,
    s sep=4mm,
},
  forked edges,
  if level=0{
    inner xsep=0pt,
    tikz={\draw [thick] (.children first) -- (.children last);}
  }{},
  [Small Scale Fadding
    [Based on Doppler Spread
      [\textbf{Fast Fadding}\\ Coherence Time$\textless $ Symbol period]
      [\textbf{Slow Fadding}\\ Coherence Time$\textgreater $ Symbol period]
      ]
  [Based on delay Spread
      [\textbf{Flat Fading}\\a)BW signal $\textless $BW channel\\b)Delay spread$\textless$Symbol Period]
      [\textbf{Frequency Selective Fadding} \\a)BW signal $\textgreater $BW channel\\b)Delay spread$\textgreater$Symbol Period]]
      ]
    ]
  ]]]
\end{forest}
}
\caption{Small scale fading}
\label{lb_sm_fadd}
\end{figure}

As discussed earlier, the small scale fading is influenced by different factors including multipath components , doppler shift and transmission bandwidth of the signal.  Based on this the small scale fading is further subdivided into based on if it is doppler or delay spread as shown in figure \ref{lb_sm_fadd}.
\subsection{Based on delay spread: }
Channel on the basis of delay spread can be subdivided into frequency selective or non selective also known as flat fading channel. In order to understand both types the knowledge of coherence bandwidth $B_c$ is crucial, since channel doesn’t behave same for all frequency ranges. The 
\textit{Coherence bandwidth} $B_c$ is the range of frequencies over which the channel is assumed to be approximately constant. 
If the signal will be received within coherent bandwidth there is no overlap with the neighbouring symbol also known as inter symbol interference (ISI) , on the other hand if the signal exceeds coherence bandwidth it will effect the neighbouring symbol causing inter symbol interference (ISI) which eventually results into fading. 
The wireless channel can be considered as frequency non-selective or flat fading if the bandwidth of signal is less than $B_c$. In this type of fading all received multipath component of a symbol arrive within the symbol time duration\cite{fading_sprger}.Therefore, there is no inter symbol interference(ISI), there will be no overlap with the neighbouring received symbols. Moreover, when the delay spread is less than symbol period in digital modulation, the affect of delay spread is insignificant on the performance of communication system. Flat fading channel is also termed as amplitude varying channel or narrow band channel.\\
The channel will experience frequency  selective fading if bandwidth of the signal is greater than $B_c$. This type of situation occurs when the multipath components of the received symbol extend beyond symbol period, resulting in overlap of a symbol with the neighbouring symbols, causing ISI.
Small scale fading is also known as \textit{Rayleigh fading}, when the there are large number of multipath components but there is no line of sight path. In this The received signal will exhibit rapid fluctuations and its envelop can be statistically described by Rayleigh distribution.\\
\subsection{Based on Doppler spread:}
The received signal power is affected with the motion of receiver or scatters with respect to transmitter, resulting in fast or slow fading.  In order to understand doppler spread based classification coherence time of channel must be understood, as both of the aforementioned fading types depend on it. Channel coherence time $T_c$ is the duration at which the channel impulse response(CIR) is assumed to be approximately constant. It is inversely proportional to the velocity and can be written as
\begin{equation}
    T_c=\frac{c}{4vf_c}
\end{equation}
where c is the speed of light and $f_c$ is the central frequency of the emitter. 
%The channel estimate in wireless system can be obtained with pilot based training, which is essential to reverse the impact of channel on the received signal or to better align the beam in the direction of user. But this estimate is only valid within this channel coherence time $T_c$.  
%The Multipath components(MPC) in small scale fading can be combined constructively or destructively.It depends on the phase of MPC at receiver.
\section{Channel Models}
In wireless communication for better system implementation, accurate and proper channel modeling is required. However, it is difficult to represent actual wireless channel accurately, as real channel is complex. Typically, the channel coefficients for all kinds of physical propagation effects(i-e path loss, shadowing, small scale fading) between transmitter and receiver are modeled by a complex random variable.\\
One possible way to distinguish individual models is the type of channel being processed,  i.e., flat fading (narrowband) vs.frequency-selective (broadband) models, time-varying vs. time-invariant models, etc. Narrowband MIMO channels can be perfectly modeled with respect to their spatial structure.

Fundamentally, MIMO channel can be classified based on physical propagation environment by using physical models (i-e ,deterministic models, geometry-based stochastic
models, and non-geometric stochastic models) or  analytical /mathematical model, which are characterised by impulse response of channel without taking into account the wave propagation. Physical model explicitly examine propagation parameters like the complex amplitude, direction of departure(DoD), direction of arrival(DoA), and delay of multipath component. However, they are independent of antenna configurations, for example, array geometry, antenna pattern, mutual coupling, polarization and the bandwidth of the system\cite{chan_model}.
In more details, in geometry-based stochastic channel models (GSCM), the impulse response is described by considering specifications of radio wave propagation for a particular transmitter-receiver pair, and scatterer geometries, which are  stochastic(random)  in nature. On the other hand, in non-geometric stochastic models, physical parameters (DoD, DoA, delay, etc.) are usually determined and represented in  a perfectly stochastic way, by taking into account the probability distribution functions without considering an underlying geometry.\\
In opposite to physical models, analytical channel models are based on individual impulse responses which are
subsumed in a (MIMO) channel matrix. These models are used for simulating MIMO matrices in the domain of system and algorithm development. Analytical models can be categorized as propagation-motivated models and correlation-based models.
In the aforementioned category, the channel matrix is modeled with propagation parameters while in the latter case MIMO channel matrix is built statistically by giving the correlation between matrix entries. Examples of propagation motive model are the maximum entropy model\cite{channel_model_entropy}, finite scatterer model\cite{channel_model_analy1},  and the virtual channel representation\cite{channel_model_virtual}. 
With analytical models, it is necessary to prescribe sets of representative parameters for the targeted scenarios. Some examples of such standardized models are 3GGP \cite{SCM}, COST 259 [12], [11], COST 273 \cite{cost273_ch_model}, IEEE 802.16a \cite{IEEE_802.16},
and IEEE 802.11n \cite{IEEE_802.11}.\\
In our simulation setups for mobile communication we have generally considered spatial channel model(SCM) standard developed by 3rd generation partnership project(3GPP) and international telecommunication union(ITU) for different propagation scenarios including urban and rural. 
The propagation waves are superimposed on the position of the antennas in the simulation configuration. Moreover, the channel coefficients are calculated based on the effect of power, delay and angular parameters at different instances of time. Furthermore, these superimposed waves produce correlation among antenna elements at both transmitter and receiver sides. We consider time-synchronized L cell with K user terminal in each cells. The BS contain $N_t$ transmit antennas and $N_r$ receive antennas. Our analytical analysis is independent of array geometry at BS.
%\section{Channel estimation}
\section{Detection and precoding }
In MIMO system precoding and detection schemes are applied to distinguish each node data streams with high level of reliability. In the downlink, the base station performs precoding to pre-filter the signal of the intended terminal. In the case of uplink, the base station does the post-filtering on the received signals from multi-users.
The purpose of the receiver is to detect the transmitted signal x from $y=Hx+n$, that is to estimate $\hat{x}$ given y and noisy channel H.\\
In the forward link or downlink, BS broadcast signal to all users in a shared channel, which can produce intra-cell interference if proper multiplexing schemes are not used. The forward link channel matrix for multi-user MIMO is $H \in N_t \times K$. To reduce the inter-user interference, BS  construct a noisy estimate of this channel matrix. This estimate is utilized by precoding techniques(discussed latter) to precode data by weighting matrix $W$.The forward link transmitted data from $j^{th}$ BS can be written as
\begin{equation}
    \mathbf{y_j}= \sqrt{\frac{\rho_{BS}}{N_t.K}} \mathbf{w_j}^T \mathbf{a}_j
\end{equation}
where precoding vector $w_j=[w_{j1},w_{j2}\hdots,w_{jK}]$ and data symbols for each terminal are  $a_j=[a_{j1} a_{j2} \hdots a_{jK}]$
 , the $\rho_{BS}$  is the total transmitted power of a BS. After passing through the channel, the signal $x_j$ transmitted by $j^{th}$ BS is received at each user terminal in $l^{th}$ cell is,
 \begin{equation}
     \mathbf{x_j}= \mathbf{H_j}^T.\mathbf{y_j}+\sum_{l=1,l \neq j} \mathbf{H_{jl}^T y_l+n_j}
     \label{fw_ueq}
 \end{equation}
 The first term in equation \ref{fw_ueq} is the desired signal with channel affects and interference(intra-cell) and second term includes inter-cell interference from other cells.\\
 In reverse link or uplink transmission, BS receive data stream from all the terminals over a multiple access channel.Then, BS decoded this multiplex signal using  spatial signatures of each user. In addition to desired signal, BS also receive the signals from neighbouring cells which acts as interference.
 \begin{equation}
    \mathbf{x}_j=\sqrt{\rho_T}\mathbf{H}_j \mathbf{a}_j+\sum_{l=1,l\neq j} \mathbf{H}_{jl} \mathbf{a}_j+\mathbf{n}_j
    \label{rev_ueq}
 \end{equation}
 In equation (\ref{rev_ueq}) the first part is desired signal from $j^{th}$ BS while second term is interfering signals from other cells.We are assuming that pilots for different users in one cell with be orthogonal to each other, so there will be almost no intra-cell interference. 
 To maximize the SNR for each link or to remove the interference between links precoding or post-processing techniques are required. Commonly used linear beamforming techniques are following.
 \subsection{Conjugate Beamforming}
 Conjugate beamforming or maximum ratio combiner/transmitter can be used to maximize SNR for each user while ignoring the impact of interference on other terminals.The precoding weight $w_j$ with conjugate beamforming can be written as,
 \begin{equation}
     \mathbf{w}_{jk}=\frac{\mathbf{\hat{h}_{jk}}^*}{\|\mathbf{\hat{H}}_j\|}
     \label{mrc_w} 
 \end{equation}
 In the above equation $*$  shows the complex conjugate and the precoding weight vector $w_{jk}$ is intended to satisfy the  power requirement at $k^{th}$ terminal. Furthermore, the power assigned to each user is normalized on each link so that the total transmission power BS remains the same. However, the users near to BS will have higher fraction of total power and they will build a stronger link.
 The drawback of conjugate beamforming is as stated earlier it ignores the effect of  multi user interference.
 \subsection{Zero-Forcing beamforming}
 Unlike conjugate beamforming, zero-forcing(ZF) precoder takes the interuser interference into account but overlooks the impact of noise. With ZF, the multiuser interference is completely cancelled by projecting each stream on the orthogonal complement of interference between users.
 The ZF weight vector can be written as
 \begin{equation}
     \mathbf{w}_{j}=
     %\frac{\mathbf{\hat{h}_{jk}}^\dagger}
     \|(\mathbf{\hat{H}}_j^H \mathbf{\hat{H}}_j)^{-1} \mathbf{\hat{H}}_j^H\|
     \label{zf_wt}
 \end{equation}
 In the ZF precoder, the weight vector streers the null values to the position of each terminal $ l^{th} (l \neq k) $ to reduce interference, transmitting everywhere else.
 The drawback of  ZF is that it works poorly in noise-limited scenarios because it neglects the effect of noise. Moreover, if the channel is noisy, the pseudo-inverse in the denominator in equation \ref{zf_wt} amplifies the noise significantly.The computational complexity of ZF is higher than conjugate beamforming due to the computation of the pseudo-inverse in the denominator.

 \subsection{Minimum Mean Square Error beamforming}
The minimum mean square error (MMSE) or regularized zero forcing (RZF)  attempts to maintain the strong signal gain while limiting interference among users. MMSE based precoding is better than ZF that gives noise amplification when user channel is highly correlated.
Going into more details, MMSE precoder is not only robust for MU-MIMO systems where users are close enough, but it also gives good performance at high SNRs\cite{precding_survay}. Indeed, it is considered as the best choice for linear precoding in MIMO based wireless systems \cite{precding_survay,precoding_emil}. For MMSE based precoding, we will write the  beamforming matrix $\mathbf{W}_j$  as follows: 
%\begin{equation}
\begin{flalign*}
& \mathbf{W}_{temp}= \mathbf{\hat{H}_j}^H (\mathbf{\hat{H}_j} \mathbf{\hat{H}_j}^H +\lambda \mathbf{I} )^{-1 } \\
& \eta= \sqrt{\frac{N_t}{Tr (\mathbf{W}_{temp} \mathbf{ W}_{temp}^H)}}\\
& \mathbf{W}_j= \eta \mathbf{ W}_{temp} \\
\label{eqMMSE}
\end{flalign*}
%\label{eqMMSE}
%\end{equation}
In the above equation $\lambda= N_t \sigma^2 /K$ is the regularization factor. When $\lambda=0$, the MMSE precoder becomes equal to a ZF one.The MMSE optimal precoder $W_{temp}$ is obtained from the estimated channel $\mathbf{\Hat{H}}$. Then, it is scaled  with the power scaling factor $\eta$ to minimize the mean square error(MSE) under the  BS transmit power constraints \cite{power_MMSE}.