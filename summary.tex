\clearpage
\begin{centering}
\textbf{SUMMARY}\\
\vspace{\baselineskip}
\end{centering}

The evolution of wireless communication brought several innovative technologies, Multiple Input Multiple Output(MIMO) system is one of such technology which gained popularity due to its capability to enhance spectral efficiency and reliability.  Although, MIMO aides enhancing system capacity and performance, however, it is challenging due to the high number of antennas at a transmitter and serving large numbers of users simultaneously.  It has been therefore the popular area of research in the last decade, meeting ever-increasing demands of data rates.   Nevertheless, severing multiple terminals simultaneously is challenging due to interference among them. The main goal of this research is to mitigate interference among users,  gain better energy and spectral efficiency by employing different DSP based algorithms using MIMO communication paradigm. \\
In this thesis, we have investigated different MIMO based research problems to enhance throughput, which is essentially achieved by mitigating inter-terminal interference by employing directional beams. To employ a directional beam it is imperative to have channel knowledge, which can be accomplished by performing channel estimation. This estimate can be achieved by using time or frequency duplexing, although, with an increased number of antennas in large scale MIMO (massive MIMO), the problem becomes more complicated in both types of duplexing schemes. The predicament can be addressed properly if the high dimensional signal is reduced to a low dimension by taking the compressive sensing(CS) paradigm into account. A framework is proposed to reduce training and feedback overhead by considering the MIMO channel as sparse in mobile communication.  Another important problem in the modern communication system is of phase recovery,  a reduced complexity Kalman filtering based solution is proposed to address the phase recovery problem in XPIC systems. Further, a novel method is devised that allows multiple implants in the intra-body network to communicate in an energy-efficient manner. 
The comparison with state of the art methods is also exhibited. The research work conducted in this thesis addresses theoretical, methodological and empirical contributions to MIMO based system research problem and attempted to achieve better performance by employing different digital signal processing(DSP) based algorithms.